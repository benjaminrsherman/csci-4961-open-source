\documentclass[11pt]{article}
\usepackage{relsize}
\usepackage[margin=1in]{geometry}
\usepackage{hyperref}
\usepackage{tabularx}
\pagenumbering{gobble}

\title{\vspace{-2.0cm}\bf Analysis of an Open Source Project}
\author{Benjamin Sherman}
\date{}

\begin{document}
\maketitle

\section{Ranking of Three Open Source Projects}
\subsection{YACS}
\nopagebreak
Website: \url{https://yacs.io}
\nopagebreak
\begin{center}
\begin{tabularx}{\textwidth}{|l|c|X|}
	\hline
	\textbf{Evaluation Factor} & \textbf{Level (0-2)} & \textbf{Evaluation Data} \\\hline
	Licensing & 2 & YACS uses the \textit{GNU Affero General Public License}, which is OSI approved.\\\hline
	Language & 1 & YACS mostly uses Ruby, which I don't really know, however there are some components which are written in Rust, which I enjoy using.\\\hline
	Level of Activity & 2 & YACS has frequent updates.\\\hline
	Number of Contributors & 2 & YACS has many contributors.\\\hline
	Product Size & 2 & There is a decently sized code base.\\\hline
	Issue Tracker & 2 & The issue tracker is being frequently used.\\\hline
	New Contributors & 2 & There is a start up guide on the site homepage.\\\hline
	Community Norms & 2 & There is a Code of Conduct which is followed.\\\hline
	User Base & 2 & It's YACS... There's a devoted user base.\\\hline
	Total Score & 17 & \\\hline
\end{tabularx}
\end{center}

%\newpage
\subsection{GIMP}
\nopagebreak
Website: \url{https://www.gimp.org/}
\nopagebreak
\begin{center}
\begin{tabularx}{\textwidth}{|l|c|X|}
	\hline
	\textbf{Evaluation Factor} & \textbf{Level (0-2)} & \textbf{Evaluation Data} \\\hline
	Licensing & 2 & GIMP uses \textit{GNU GPLv3}, which is an OSI approved license.\\\hline
	Language & 2 & GIMP is written in C, which is a language I enjoy.\\\hline
	Level of Activity & 2 & GIMP has frequent code updates and releases.\\\hline
	Number of Contributors & 2 & GIMP has nearly 100 contributors.\\\hline
	Product Size & 1 & The GIMP code base is very complex.\\\hline
	Issue Tracker & 2 & The GIMP issue tracker is widely used.\\\hline
	New Contributors & 2 & There are multiple files and a wiki dedicated to helping developers, and many of these resources cater to new developers.\\\hline
	Community Norms & 1 & GIMP mailing lists have a code of conduct, but it is unclear whether or not those rules apply beyond the mailing lists.  Regardless, the community is respectful.\\\hline
	User Base & 2 & It was estimated that in 2014 there were 100,000 downloads of GIMP by Ubuntu derivatives alone.\footnote{\url{https://www.gimpusers.com/forums/gimp-user/16238-how-manu-gimp-user-are-there}}\\\hline
	Total Score & 16 & \\\hline
\end{tabularx}
\end{center}

\subsection{The Linux Kernel}
\nopagebreak
Website: \url{https://www.kernel.org/}
\nopagebreak
\begin{center}
\begin{tabularx}{\textwidth}{|l|c|X|}
	\hline
	\textbf{Evaluation Factor} & \textbf{Level (0-2)} & \textbf{Evaluation Data} \\\hline
	Licensing & 2 & The Linux Kernel is licensed under \textit{GNU GPLv2}, which is OSI approved.\\\hline
	Language & 2 & The Linux Kernel uses mostly C which is a language I enjoy.\\\hline
	Level of Activity & 2 & As of 2017, there were 8.5 changes to the kernel every hour\footnote{\url{https://www.linuxfoundation.org/publications/2017/10/2017-state-of-linux-kernel-development/}}, however that rate has increased since.\\\hline
	Number of Contributors & 2 & As of 2017, there were over 1,600 kernel developers with over 15,637 total developers since 2005.\footnote{\url{https://www.linuxfoundation.org/publications/2017/10/2017-state-of-linux-kernel-development/}}\\\hline
	Product Size & 2 & The Linux kernel's code base is very large, however this does not cause much of an issue.  See \ref{subsec:kernel_codebase} for more information.\\\hline
	Issue Tracker & 2 & The Linux kernel uses mailing lists to organize development and issue tracking, and they are \textit{very} active.\\\hline
	New Contributors & 2 & While a high level of knowledge is needed to contribute to the kernel's core, it is surprisingly easy to submit small fixes and there are many resources dedicating to helping newcomers learn about kernel development.\\\hline
	Community Norms & 2 & Overall, the kernel community is very respectful.  See \ref{subsec:kernel_community} for more information.\\\hline
	User Base & 2 & The Linux kernel has many users.  See \ref{subsec:kernel_users} for more information.\\\hline
	Total Score & 17 & \\\hline
\end{tabularx}
\end{center}

\section{In-Depth Analysis: The Linux Kernel}
\subsection{Licensing}
\label{subsec:kernel_license}

\subsection{Language}
\label{subsec:kernel_language}

\subsection{Level of Activity}
\label{subsec:kernel_activity}

\subsection{Number of Contributors}
\label{subsec:kernel_contributors}

\subsection{Product Size}
\label{subsec:kernel_codebase}

\subsection{Issue Tracker}
\label{subsec:kernel_issue_tracker}

\subsection{New Contributors}
\label{subsec:kernel_newbies}

\subsection{Community Norms}
\label{subsec:kernel_community}

\subsection{User Base}
\label{subsec:kernel_users}

\end{document}
